For the implementation of the program, several algorithms have been implemented
(and combined in various ways). First, the algorithms will be described, then
the implementations of the algorithms will discussed.

\subsection{Algorithms}
\label{sec:algorithms}

\subsubsection{Primality testing}
The input can contains prime numbers, and for these number, no factors can be
found since they are prime. Therefore, before the input is factored, the input
needs to be tested to see if it is prime. 
This was done by using the Miller-Rabin
algorithm~\cite{wiki:miller-rabin}.

\subsubsection{General algorithm}
The task of factoring a given number $n$ can be split into three tasks:
\begin{itemize}
    \item Find a factor $m$ of $n$
    \item Find the exponent $k$ of factor $m$
    \item Print the number $m$ $k$ times
\end{itemize}
Therefore, the task of finding a prime factor $m$ of $n$ can be implemented in a
subroutine \emph{find-prime-factor}, resulting in algorithm \ref{alg:generic}.

\begin{pseudocode}{factor}{n}
\label{alg:generic}
\COMMENT{General algorithm for factoring a number $n$} \\
    \WHILE n \neq 1 \DO
        \BEGIN
            m \GETS \textit{find-prime-factor}(n) \\
            k \GETS 1 \\
            \WHILE m \mid n \DO
                \BEGIN
                    k \GETS k + 1 \\
                    n \GETS $m / n$
                \END \\
            \FOR i \GETS 0 \TO k \DO 
                \textit{print}(m)
        \END \\
\end{pseudocode}

\subsubsection{Trial division}
\label{sec:trial-division}
If an integer $n$ is composite (not prime), then $n$ must have a factor less
than or equal to $\sqrt n$, since $\sqrt n \cdot \sqrt n = n$. Therefore, a
naive algorithm for finding a prime factor to $n$ is to try all primes
less than or equal to $\sqrt n$ and check if any of them divides $n$. This
yields algorithm \ref{alg:naive}.

\begin{pseudocode}{trial-division}{n}
\label{alg:naive}
\FOR i \GETS 2 \TO \sqrt n \DO
    \BEGIN
        \IF i \mid n \THEN \RETURN i \\
        i \GETS \textit{next-prime}(i)
    \END
\end{pseudocode}

\subsubsection{Pollard's $\rho$ algorithm}
\label{sec:pollard-rho}
In 1975, Jonh Pollard invented an algorithm that today is referred to as
Pollard's $\rho$ algorithm~\cite{wiki:pollard}. Pollard's $\rho$ algorithm
tries to find two numbers $x$ and $y$ such that $x \nmid n$ and $y \nmid n$, 
but $(x - y) \mid n$. Since $n$ is composite, $n = pq$ for some $p$,
$q$ where $gcd(p,q) = 1$.  By the Chinese Remainder Theorem~\cite{wiki:crt}, 
$x \bmod n$ can be
represented by the pair $(x \bmod p, x \bmod q)$. When $x \equiv y \bmod p$,
then
$p \mid (x - y)$ and since $p \mid n$, $(x - y) \mid n$ and the only factor 
$(x - y)$ and $n$ can share is $p$. Hence, $p = gcd(x-y, n)$. 
Therefore, when two
numbers $x$ and $y$ are found such that $gcd(x - y, n) \neq 1$, a factor $p$
is found. Therefore, Pollard's $\rho$ algorithm needs to generate numbers such
that $x \not \equiv y \bmod n$ and $x \equiv y \bmod p$.

Iterating a polynomial formula $f(x) = x^2 + c \; (\bmod \; n)$ will yield 
numbers that might turn into a cycle, that is, the returned values will be
repeated. The possibility that $g(x) = x^2  + c \; (\bmod \; p)$ 
will turn into a
cycle is higher, since $p < n$. Therefore, by repeatedly applying $f(x)$ to
some start value $x_0$, two numbers $x$ and $y$ such that 
$x = y \; (\bmod \; p)$ will be 
found if the function $f(x)$ enters a cycle. Then, since $p \leq \sqrt n$,
there is a high probability that $x \not \equiv \bmod
n$.
The following two paragraphs describes two different ways for
finding such a cycle.

\paragraph{Floyd's cycle detection}
Floyd's cycle detection is based on the idea of using two values $x$ and $y$
and for then repeatedly performing $x = f(x)$ and $y = f(f(y))$, that is, $y$
will ''move'' twice as fast. If a cycle is found, then $y$ will enter the cycle
and then ''catch up'' with $x$ when $x$ enters the cycle. 
However, there is a change that $x = y = n$, and in this case, the algorithm
fails.

Using Floyd's cycle
detection results in algorithm \ref{alg:floyd}.

\begin{pseudocode}{pollard-floyd}{n}
\label{alg:floyd}
d \GETS 1 \\
\WHILE d = 1 \DO
    \BEGIN
        x \GETS f(x) \\
        y \GETS f(f(y)) \\
        d \GETS gcd(|y-x|, n)
    \END \\
\IF d = n \THEN \RETURN fail \ELSE \RETURN d
\end{pseudocode}

\paragraph{Brent's cycle detection}
Brent's cycle detection is similar to Floyd's, but differ in how often $y$ is
updated. Instead of updating $y$ at every iteration, $y$ is updated whenever
the number of iterations is a power of 2.

Using Brent's cycle detection results in algorithm \ref{alg:brent}.

\begin{pseudocode}{pollard-brent}{n}
\label{alg:brent}
d \GETS 1 \\
c \GETS 0 \\
\WHILE d = 1 \DO
    \BEGIN
        x \GETS f(x) \\
        \IF c \equiv 0 \bmod 2 \THEN
            y = x \\
        d \GETS gcd(|y-x|, n) \\
        c \GETS c + 1
    \END \\
\IF d = n \THEN \RETURN fail \ELSE \RETURN d
\end{pseudocode}

Note that the Pollard $\rho$ algorithm does \emph{not} find a prime factor of
$n$, it justs a finds a factor. Therefore, in order to find a prime factor, the
algorithm can be reapplied to the found factor in a recursive step. This results
in algorithm \ref{alg:pollard-prime}.

\begin{pseudocode}{pollard-prime}{n}
\label{alg:pollard-prime}
d \GETS \textit{pollard}(n) \\
\WHILE \textit{is-prime}(d) = false \DO
    d \GETS \textit{pollard}(d) \\
\RETURN d
\end{pseudocode}

\subsubsection{Perfect power factorization}
Some composite numbers are perfect powers, that is, an integer $n = p^k$ for
some prime $p$ and some integer $k$. Such numbers can be factored by using
Newtons method~\cite{wiki:newton} by repeatedly trying to take $j$th root of
$n$ and check if it yields an integer for different integers $j$.
Since $n$ is a perfect power, $j$ has to be larger than 2.
$j$ has to be less than $\lceil log_2(n) \rceil$, 
since $p \geq 2$. This results in algorithm
\ref{alg:perfect-power}.

\begin{pseudocode}{perfect-power-factorization}{n}
\label{alg:perfect-power}
j_{max} \GETS \lceil log_2(n) \rceil \\
\FOR j \GETS j_{max} \TO 2 \DO
\BEGIN
    r \GETS \textit{newton}(n,j) \\
    \IF r \in \mathbb{Z} \THEN
        \RETURN r
\END
\end{pseudocode}

\subsection{Implementation of algorithms}
\label{sec:implementation-of-algorithms}
All algorithms were implemented in the C programming language using the
GMP~\cite{gmp}
library for all arithmetic operations involving large integers.

The different parts of the implementation will be discussed in
the following sections. 

\subsubsection{Handling failure}
\label{sec:handling-failure}

Since it might be the case that the program won't be able to factor all the
given numbers and an answer has to be given for all
numbers, a way was needed to abort the factorization of a number.

\paragraph{Handling time-outs}
\label{sec:time-outs}

Since the program isn't allowed to use the \texttt{time} system call, 
this was implemented by using a cut-off value. Every iteration of the factoring
algorithm decremented a counter with an initial max value. If the counter
reached 0, that meant that the number couldn't be factored.

\paragraph{Handling partial factorization}
Since the program could find some of the factors, but not all of them due to a
time-out, the factors couldn't be printed directly. Instead, they were stored
in a pre-allocated array, and if the algorithm succeeded, the contents of the
array was printed.

\subsubsection{Implementing primality testing}
The GMP \texttt{mpz\_probab\_prime\_p} is using the Miller-Rabin test to test
if a given number $n$ is prime or not. \texttt{mpz\_probab\_prime\_p} also
takes a second parameter \texttt{reps} describing how many times Miller-Rabin
should be run to increase the accuracy of the test. We used a value of 3.

\subsubsection{Implementing trial division}
Trial division was implemented by testing a fixed number of primes stored 
in an array. The array was iterated over in a loop, trying to divide the given
number $n$ with each prime $p$ in the array.

\subsubsection{Implementing Pollard's $\rho$ algorithm}
The implementation of Pollard's $\rho$ algorithm followed algorithm
\ref{alg:floyd} closely. 

The algorithm used the polynomial function $f(x) = x^2 + 2 \; (\bmod n)$ as for
generating pseudo-random numbers with the initial value of $x$ set to 1.

When implementing the algorithm, the case
when the algorithm fails due to $x = y = n$  has to be handled. 
The way this was done was to restart the algorithm,
but with a different polynomial function $g(x) = x^2 + c \; (\bmod n)$ and a 
new initial value $i$. The values $i$ and $c$ were chosen randomly when
restarting.

\subsubsection{Implementing perfect power factorization}
The implementation of perfect power hashing made use of the GMP function
\texttt{mpz\_root} for finding the n:th root, even though using this 
function was probably against the
rules. The reason for this was to benchmark a solution making use of perfect
power factorization, and if a performance increase noticed, 
implement Newton's method
ourself. The implementation followed algorithm \ref{alg:perfect-power} closely.
