Given 100 unknown integers, compute and print the prime number factorization 
for each integer. If number $n_i = p_1^{k_1}p_2^{k_2} \ldots p_m^{k_m}$, then
the program should print the following on \texttt{stdout}.
\begin{align*}
&\left.\begin{aligned}
      &p_1 \\
      &p_1 \\
      &\vdots \\
      &p_1 
      \end{aligned}
\right\}
\qquad k_1 \text{ times} \\
&\left.\begin{aligned}
      &p_2 \\
      &p_2 \\
      &\vdots \\
      &p_2 
      \end{aligned}
\right\}
\qquad k_2 \text{ times} \\
&\vdots \\
&\left.\begin{aligned}
      &p_m \\
      &p_m \\
      &\vdots \\
      &p_m 
      \end{aligned}
\right\}
\qquad k_m \text{ times}
\end{align*}

The time limit for the program is 15 seconds. The text \texttt{fail} must be
printed on \texttt{stdout} if the program fails to factor a number. 
The performance of the program is measured in how many numbers the program are
able to factor within the time limit. The program must print either the correct
prime factorization or \texttt{fail} for each of the 100 numbers.

The program is \emph{not} allowed to use the \texttt{time} system call.
