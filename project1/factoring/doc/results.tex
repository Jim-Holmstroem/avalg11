\subsection{Benchmark}
To benchmark the algorithm we measured the execution time for 100 randomly sampled 
semiprimes $p*q=n$ with a bitlength of $p,q=(u,v)$.
The factorization is symmetric on $p,q$, therefore we only did measurements 
where $u \le v$.
The values $(u,v) \in 4[1,12]^2$ where used in the tests.

We also did some comparisons between some algorithms and their parameters by 
directly getting points from \texttt{KATTIS}~\cite{kattis} on the problem 
\texttt{oldkattis:factoring}~\cite{factoring}.

\subsection{Test system}
All measurements was made on a computer using Ubuntu 10.04 LTS (64-bit)
on a Intel Xeon X3470 CPU @ 2.93GHz with 4 cores (8 threads) and with 16GB RAM. 
The code was compiled with \texttt{gcc-4.4.3} using the flag \texttt{-g} and 
the library \texttt{GMP-4.3.2}.

The execution time was measured with the \texttt{USER} time from the 
\texttt{time} command.

\subsection{Test results}
These tests were solely based on the score and running time on  the 
\texttt{KATTIS} problem \texttt{oldkattis:factoring}.

\paragraph{Primality testing}
Using only the Miller-Rabin tests do determine if a number is prime and
if so print the number. The results can be seen in table
\ref{table:miller-rabin}

\begin{table}[h!]
    \centering
    \begin{tabular}{|c|c|}
    \hline
    \textbf{Score} & \textbf{Running time} \\ \hline
    1              & 0.01                  \\ \hline
    \end{tabular}
    \caption{Only using the Miller-Rabin test}
    \label{table:miller-rabin}
\end{table}

\paragraph{Trial division and primality test}
The next version uses trial division (as described in algorithm
\ref{alg:naive}) and the Miller-Rabin primality test. A fixed number of primes were
used for trial division, but the number of primes we used varied. 
The results can be seen in table \ref{table:trial}.

\begin{table}[h!]
    \centering
    \begin{tabular}{|c|c|c|}
    \hline
    \textbf{Number of primes} & \textbf{Score} & \textbf{Running time} \\ \hline
    10000 & 21 & 0.05 \\ \hline
    7500  & 21 & 0.05 \\ \hline
    5000  & 21 & 0.04 \\ \hline
    2500  & 21 & 0.02 \\ \hline
    1875  & 21 & 0.02 \\ \hline
    1718  & 21 & 0.02 \\ \hline 
    1562  & 20 & 0.02 \\ \hline 
    1250  & 19 & 0.02 \\ \hline
    625   & 18 & 0.01 \\ \hline
    312   & 18 & 0.01 \\ \hline
    156   & 15 & 0.01 \\ \hline
    78    & 14 & 0.01 \\ \hline
    39    & 13 & 0.01 \\ \hline
    19    & 9  & 0.01 \\ \hline
    10    & 7  & 0.01 \\ \hline
    5     & 4  & 0.01 \\ \hline
    0     & 1  & 0.01 \\ \hline
    \end{tabular}
    \caption{Using trial division and primality test}
    \label{table:trial}
\end{table}


\paragraph{Pollard's $\rho$}
Pollard's $\rho$ algorithm was tested in a few different ways. The different
combinations used were:
\begin{enumerate}[A)]
    \item Only Pollard's $\rho$ with Floyd's cycle detection and primality test
    \item Only Pollard's $\rho$ with Brent's cycle detection and primality test
    \item Pollard's $\rho$ with Brent's cycle detection, trial division and 
          primality test
    \item Pollard's $\rho$ with Floyd's cycle detection, handling perfect
          powers, trial division and primality test
\end{enumerate}
The results can be seen in table \ref{table:pollard-comparison}.

\begin{table}[h!]
    \centering
    \begin{tabular}{|c|c|c|}
    \hline
    \textbf{Variant} & \textbf{Score} \\ \hline
    A                & 72             \\ \hline
    B                & 64             \\ \hline
    C                & 71             \\ \hline
    D                & 72             \\ \hline
    \end{tabular}
    \caption{Comparing different implementations Pollard's $\rho$ algorithm}
    \label{table:pollard-comparison}
\end{table}

Given that Floyd's cycle detection seemed superior to Brent's cycle detection,
we, Pollard's $\rho$ algorithm with Floyd's cycle detection was tested with 
two different random value generators. 
The first one used was GMPs \texttt{mp\_urandomm} and the second
being used was \texttt{rand}. The results can be seen in table
\ref{table:random}.

\begin{table}[h!]
    \centering
    \begin{tabular}{|c|c|}
    \hline
    \textbf{Variant}     & \textbf{Score} \\ \hline
    \texttt{mp\_urandomm} & 72             \\ \hline
    \texttt{rand}        & 75             \\ \hline
    \end{tabular}
    \caption{Comparing different random generators}
    \label{table:random}
\end{table}

\paragraph{Algorithm cut-off}
As a final test, Pollard's $\rho$ algorithm was tested with different cut-off
values and the result can be seen in table \ref{table:cutoff} (see section for
\ref{sec:time-outs} more details about the cut-off value).

\emph{Note:} this algorithm used \texttt{mp\_urandomm}, therefore the maximum
score was 72.

\begin{table}[h!]
    \centering
    \begin{tabular}{|c|c|c|}
    \hline
    \textbf{Cut-off value} & \textbf{Score} & \textbf{Running time} \\ \hline
    190000 & 72 & 14.32 \\ \hline
    172500 & 71 & 13.05 \\ \hline
    155000 & 71 & 11.93 \\ \hline
    137500 & 71 & 10.97 \\ \hline
    120000 & 70 & 10.01 \\ \hline
    110000 & 70 & 9.50 \\ \hline
    100000 & 69 & 8.70 \\ \hline
    90000  & 66 & 8.21 \\ \hline
    80000  & 65 & 7.48 \\ \hline 
    70000  & 65 & 6.77 \\ \hline   
    60000  & 63 & 6.00 \\ \hline
    50000  & 58 & 5.22 \\ \hline
    40000  & 56 & 4.28 \\ \hline
    35000  & 55 & 3.83 \\ \hline
    30000  & 50 & 3.35 \\ \hline
    25000  & 49 & 2.86 \\ \hline
    20000  & 44 & 2.37 \\ \hline
    15000  & 41 & 1.82 \\ \hline
    10000  & 38 & 1.25 \\ \hline
    5000   & 35 & 0.67 \\ \hline
    0      & 1  & 0.01 \\ \hline
    \end{tabular}
    \caption{Comparing different cut-off values}
    \label{table:cutoff}
\end{table}

\subsection{Kattis submission}
The best Kattis submission has the id 256166.
