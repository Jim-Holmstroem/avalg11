\documentclass[a4paper,twoside=false,abstract=false,numbers=noenddot,
titlepage=false,headings=small,parskip=half,version=last]{scrartcl}

\usepackage[utf8]{inputenc}
\usepackage[T1]{fontenc}
\usepackage[english]{babel}

\usepackage[colorlinks=true, pdfstartview=FitV,
linkcolor=black, citecolor=black, urlcolor=blue]{hyperref}
\usepackage{verbatim}
\usepackage{graphicx}
\usepackage{multirow}

\usepackage{amsmath}
\usepackage{amsthm}
\usepackage{amssymb}
\usepackage{amsfonts}

\theoremstyle{definition}
\newtheorem{exercise}{Exercise}

\theoremstyle{remark}
\newtheorem*{solution}{Solution}
\newtheorem*{remark}{Remark}

\newcommand{\ZZ}{\ensuremath{\mathbb{Z}}}
\newcommand{\QQ}{\ensuremath{\mathbb{Q}}}
\newcommand{\RR}{\ensuremath{\mathbb{R}}}
\newcommand{\CC}{\ensuremath{\mathbb{C}}}

\DeclareMathOperator{\Hom}{Hom}
\DeclareMathOperator{\End}{End}
\DeclareMathOperator{\im}{im}
\DeclareMathOperator{\id}{id}

\renewcommand{\labelenumi}{(\alph{enumi})}

\author{Jim Holmstr�m - 890503-7571}
\title{Avancerade algoritmer DD2440}
\subtitle{Homework B}

\begin{document}

\maketitle
\thispagestyle{empty}

%-----------------------
\begin{exercise}
Using the 0/1-principle for sorting networks, either show that the following
sorting network correctly sorts 4 inputs, or find an input where it fails.
\end{exercise}
\begin{solution}
0/1-principle states that it will correctly sort any input if it can sort all
possible sequences of ${0,1}$. This reduces the combination of input to check
in an $n$-wire network from $n!$ to $2^n$.\\
In our case we will need to validate $2^4=16$ combinations.

The comparators will only swap values if they are misplaced, thus all
combinations already sorted will come out the same way (sorted).\\

The comparators only swaps values so $\sum{x}=\sum{x'}$ (where $\bar{x}=$input
and $\bar{x}'=$output)

$\forall x : \sum{x}=0,4$\\
Trivially $\bar{x}=0000$ and $\bar{x}=1111$ will still be sorted.\\
$\forall x : \sum{x}=1$\\
The one set input-value will "bubble" down from position 1,2 and 3 (and already be
sorted for position 4).\\ 
$\forall x : \sum{x}=3$\\
By symmetry this also works for the one unset variable in the same way as
above.
$\forall x : \sum{x}=2$\\
    $\bar{x}=0011$ is already sorted. Those 5 left is put in an matrix for
    batch calculations, if some rows become equal after a pass only one of them is
    considered since they will all act the same from there on.
    
    \begin{equation*}
        \tilde{X}_2= 
        \left(
        \begin{matrix}
            
                                     0 & 0 & 1 & 1 & 1 \\
                                     1 & 1 & 0 & 0 & 1 \\
                                     0 & 1 & 0 & 1 & 0 \\
                                     1 & 0 & 1 & 0 & 0
       \end{matrix}
       \right)
    \end{equation*}
    
    \begin{equation*} 
        \left(
        \begin{matrix}
            
                                     0 & 0 & 1 & 1 & 1 \\
                                     1 & 1 & 0 & 0 & 1 \\
                                     0 & 1 & 0 & 1 & 0 \\
                                     1 & 0 & 1 & 0 & 0
       \end{matrix}
       \right) \\
        = \{\mbox{1-pass: compare 1-2 and 3-4}\} =
        \left(
        \begin{matrix}
            
                                     0 & 0 & 0 & 0 & 1 \\
                                     1 & 1 & 1 & 1 & 1 \\
                                     0 & 0 & 0 & 0 & 0 \\
                                     1 & 1 & 1 & 1 & 0
       \end{matrix}
       \right) \\
        =
        \left(
        \begin{matrix}
            
                                     0 & 1 \\
                                     1 & 1 \\
                                     0 & 0 \\
                                     1 & 0
       \end{matrix}
       \right)
       =
    \end{equation*}
    \begin{equation*}
       = \{\mbox{2-pass: compare 2-3}\} =
        \left(
        \begin{matrix}
            
                                     0 & 1 \\
                                     0 & 0 \\
                                     1 & 1 \\
                                     1 & 0
       \end{matrix}
       \right)
        = \{\mbox{3-pass: compare 1-2 and 3-4}\} =
        \left(
        \begin{matrix}
            
                                     0 & 0 \\
                                     0 & 1 \\
                                     1 & 0 \\
                                     1 & 1
       \end{matrix}
       \right)
        =
    \end{equation*}
    \begin{equation*}
        =\{\mbox{4-pass: compare 2-3}\} =
        \left(
        \begin{matrix}
            
                                     0 & 0 \\
                                     0 & 0 \\
                                     1 & 1 \\
                                     1 & 1
       \end{matrix}
       \right)
    \end{equation*}

Now we have proven that all 16 combinations of binary input will be sort and by
the 0/1-principle the net will be a valid sorting net. $\qed$

\end{solution}
%-----------------------
\begin{exercise}
Crack RSA! Given the public RSA key $(n,e)=(5192042249411,3419183406883)$, find
the decryption exponent $d$ and find a number $x$ that encrypts to your
ten-digit personal number. If you wish, you may use big integer support (but
not functions specifically tailored for RSA).
\end{exercise}
\begin{solution}
Firstly we factor the semiprime $n=pq$ using a simple implementation of
Pollard-rho with $f(x)=x^2+c$ picking $c=1$

\begin{verbatim}
    function pollard_rho(n):
        d = 1,x = (f(x0)%n,f(f(x0)%n)
        while d==1
            x = (f(x[0])%n,f(f(x[1])%n)
            d = gcd(abs(x[1]-x[0]),n)
        return (d,n/d)
\end{verbatim}

We get $n=5192042244911=1532371 \cdot 3388241 = pq$

\begin{equation*}
    \varphi(n)=\varphi(pq)=\{\mbox{p,q primes}\}=(p-1)(q-1) = 5192037328800
\end{equation*}
And the private key $d$ is:
\begin{equation*}
    d=e^{-1} \ (mod \ \varphi(n))
\end{equation*}
\begin{equation*}
    ed=1 \ (mod \ \varphi(n))
\end{equation*}
\begin{equation*}
    ed - K\varphi(n)=1
\end{equation*}

$d$ is then calculated by the extended euclidean algorithm to be
\underline{$d=1456182194347$} and one can verify that it is indeed a modular inverse to
$e$. (K is irrelevant)

\begin{equation} \label{eq:encrypt}
    c=m^e \ (mod \ n)
\end{equation}
\begin{equation} \label{eq:decrypt}
    m=c^d \ (mod \ n)
\end{equation}

Since we are dealing with large exponents one cannot use the naive exponentiation 
instead we use modular exponentiation by repeated squaring to bring down the
complexity. The main idea is this:

%\begin{equation}
%    x^n = 
%    \begin{cases}
%    1                      & n=0 \\
%    x(x^{\frac{n-1}{2}})^2 & n \ \mbox{odd} \\
%    (x^{\frac{n}{2}})^2    & n \ \mbox{even}
%    \end{cases}
%\end{equation}

\begin{equation}
    a^{2n} \equiv (a^n)^2 \ ( mod \ m)
\end{equation}
\begin{equation}
    a^{2n+1} \equiv a^1(a^n)^2 \ ( mod \ m )
\end{equation}

An implementation:
\begin{verbatim}
function superpower(x,n,m):
    ans = 1
    while n: //int2bin
        n_bits.append(n%2)
        n>>=1
    n_bits.reverse()

    for bit in n_bits:
        ans = (ans*ans)%m 
        if bit: #"odd"
            ans = (ans*x)%m
    return ans
\end{verbatim}

To find the number $x$ that encrypts to my personal number $(8905037571)$ we
need to go the other way by decrypting. \\
$m=c^d \ (mod \ n) =\
$\verb+superpower+$(c,d,n)=\{c=8905037571\}=\underline{2534986786188}$

\end{solution}
%-----------------------
\begin{exercise}
Multiply the following polynomials using Karatsuba's algorithm.\\
$x^3+3x^2+x$ and $2x^3-x^2+3$.\\
You only need to show the top-most call, what recursive calls are made from the
top-most level, what those calls return and how the final result is computed.

\end{exercise}
\begin{solution}

Karatsuba's algorithm is used as base.
\begin{equation} \label{eq:karatsuba}
\begin{split}    
    (ax+b)(cx+d) = \underbrace{(a \cdot c)}_{=u}x^2+\underbrace{(a \cdot d+b
    \cdot c)}_{=w-u-v}x+\underbrace{(b \cdot d)}_{=v} \\
    \begin{cases}
        u=a \cdot c \\
        v=b \cdot d \\
        w=(a+b)\cdot (c+d) = ac+bc+ad+bd
    \end{cases}
\end{split}
\end{equation}

All higher order polynoms can be broken up this way:

\begin{equation} \label{eq:polysplit}
\begin{split}
    p(x) = & \sum_0^{2t-1}p_ix^i
    =\underbrace{\sum_0^{t-1}p_ix^i}_{b(x)}
    +x^t\underbrace{\sum_0^{t-1}p_{t+i}x^i}_{a(x)}
    =a(x)x^t+b(x) \\
    q(x) = & \{\mbox{The same way as above}\} = c(x)x^t+d(x)
\end{split}
\end{equation}
Then taking $p(x)q(x)$ yields:
\begin{equation}
\begin{split}
    p(x)q(x) = & (a(x)x^t+b(x))(c(x)x^t+d(x))= \{x^t \leftrightarrow x \ ,\mbox{
    Karatsuba's algorithm}\} = \\
    = & \underbrace{(a(x)c(x))}_{=u(x)}x^{2t}
    +\underbrace{(a(x)d(x)+b(x)c(x))}_{=w(x)-u(x)-v(x)}x^t
    +\underbrace{(b(x)d(x))}_{=v(x)}
\end{split}
\end{equation}

Our polynoms $(x^3+3x^2+x-1)(2x^3-x^2+3)$ is represented as $(1,3,1,-1)$ and
$(2,-1,0,3)$ in memory, $x^i$'s coefficient is the variable at position $i$.
All operations like addition and subtraction can be done termwise.

First split the polynoms in the middle (trivial if one looks at the memory
representation):
\begin{equation}
    (x^3+3x^2+x-1)(2x^3-x^2+3)=\{(\ref{eq:polysplit})\}=
    \underline{((x+3)x^2+(x-1))((2x-1)x^2+0x+3)}
\end{equation}

Then executing Karatsuba's algorithm
\begin{equation*}
    \begin{cases}
        u(x) = (x+3)(2x-1)=\{\mbox{Recursive call to Karatsuba's algorithm}\} =
        \underline{2x^2+5x-3} \\
        v(x) = (x-1)(0x+3)= \{\mbox{Recursive call to Karatsuba's algorithm}\}
        =\underline{3x-3} \\ 
        w(x) = (2x+2)(2x+2)=\{\mbox{Recursive call to Karatsuba's algorithm}\} =
        \underline{4x^2+8x+4}
    \end{cases}
\end{equation*}

Karatsuba's algorithm then returns\\
    $(a(x)x^2+b(x))(c(x)x^2+d(x))= \{(\ref{eq:karatsuba})\}
    = u(x)x^4+(w(x)-u(x)-v(x))x^2+v(x) \\
    = (2x^2+5x-3)x^4+(2x^2+10)x^2+(3x-3) =\underline{2x^6+5x^5-x^4+10x^2+3x-3}$

\end{solution}
%-----------------------
\begin{exercise}
A positive odd integer n is a Carmichael number if
\begin{enumerate}
\item
it is composite
\item
$\forall$ numbers $1 \le a < n : gcd(a,n)=1$, it holds that $a^{n-1}\equiv 1
    \ (mod\ n)$
\end{enumerate}
Show that it is easy to find a non-trivial factor of a Carmichael number. More
precisely, there is a polynomial-time algorithm for finding a non-trivial
factor with probability $> 1/2$.\\
Explain your algorithm, why it works, and analyze its complexity.
\end{exercise}
\begin{solution}

\end{solution}



\end{document}

