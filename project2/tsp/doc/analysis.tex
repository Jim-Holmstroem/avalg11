As can be seen in tables \ref{table:nn} and \ref{table:nn_and_2-opt}, adding an
local optimization step to the heuristic creates a much better result and the resulting score for 
\proc{3-opt} further confirms it, each added permutation type improved the path of.

It is also interesting to note that the score doesn't increase when doubling
the counter in the last row of table \ref{table:nn_and_2-opt}. This means that
\proc{2-opt} has encountered a local miminum and won't find a better solution,
no matter how much more time it gets. One can also see that the number of
iterations of \proc{2-opt} wasn't the worst case scenario of
$\Theta(2^{n/2})$, since the running time isn't more than 0.38 seconds. This
further justifies the claim of \cite{hastad} that the number of iterations in
practice are $O(n)$.

As for the \proc{3-opt} it, for most instances, didn't have time to fully complete 
and was prematuraly forced to return and would perform much better if given enough time. 
Since it rarely reached a complete run it's hard to tell if the number of re-runs was $O(n)$ or the worstcase exponential runningtime.

Adding the second permutation gave almost twice as large impact on the score as just adding the first one even though it had a higher running-complexity,
this is probably due to a simpler (discarding the reversal) permutation has a higher probability of occurence.
