\documentclass[a4paper,twoside=false,abstract=false,numbers=noenddot,
titlepage=false,headings=small,parskip=half,version=last]{scrartcl}

\usepackage[utf8]{inputenc}
\usepackage[T1]{fontenc}
\usepackage[english]{babel}

\usepackage[colorlinks=true, pdfstartview=FitV,
linkcolor=black, citecolor=black, urlcolor=blue]{hyperref}
\usepackage{verbatim}
\usepackage{graphicx}
\usepackage{multirow}

\usepackage{tikz}
\usetikzlibrary{matrix}

\usepackage{amsmath}
\usepackage{amsthm}
\usepackage{amssymb}
\usepackage{amsfonts}

\theoremstyle{definition}
\newtheorem{exercise}{Exercise}

\theoremstyle{remark}
\newtheorem*{solution}{Solution}
\newtheorem*{remark}{Remark}

\newtheorem{theorem}{Theorem}[section]
\newtheorem{lemma}[theorem]{Lemma}
\newtheorem{proposition}[theorem]{Proposition}
\newtheorem{corollary}[theorem]{Corollary}

\newcommand{\NN}{\ensuremath{\mathbb{N}}}
\newcommand{\ZZ}{\ensuremath{\mathbb{Z}}}
\newcommand{\QQ}{\ensuremath{\mathbb{Q}}}
\newcommand{\RR}{\ensuremath{\mathbb{R}}}
\newcommand{\CC}{\ensuremath{\mathbb{C}}}
\newcommand{\Fourier}{\ensuremath{\mathcal{F}}}
\newcommand{\Laplace}{\ensuremath{\mathcal{L}}}


\DeclareMathOperator{\Hom}{Hom}
\DeclareMathOperator{\End}{End}
\DeclareMathOperator{\im}{im}
\DeclareMathOperator{\id}{id}


\renewcommand{\labelenumi}{(\alph{enumi})}

\author{Jim Holmström - 890503-7571}
\title{Avancerade algoritmer DD2440}
\subtitle{Homework C}

\begin{document}

\maketitle
\thispagestyle{empty}

%-----------------------
\begin{exercise}
{\bf
Multiply the following two polynomials using FFT.

\begin{equation*}
p(x)=x^3+3x^2+x-1
\end{equation*}
\begin{equation*}
q(x)=2x^3-x^2+3
\end{equation*}

If you use a recursive version of FFT, you only need
to show the top-most call to FFT (and FFT inverse),
what recursive calls are made from the top-most level,
what those calls return and how the final result is computed.
}
\end{exercise}
\begin{solution}
Represent the polynomails $p,q$ as vectors of their coefficients where the constant coefficient is the first element $p=(-1,1,3,1,0,0,0,0)$ and $q=(3,0,-1,2,0,0,0,0)$


\end{solution}
%\clearpage
%-----------------------
\begin{exercise}
{\bf
Let $A = \{ a_{i,j} : a_{i,j} = a_{i-1,j-1} \bigwedge a_{i,1} = a_{i-1,n} \}$
for $i,j \in [2,n]$. 
Give an $O(n log(n))$-time algorithm for computing $Ax$ where $x$ is a vector of length $n$
}
\end{exercise}
\begin{solution}
Note that all indices in the following equations are beginning at $0$ and taken modulo $n$ 
The matrix $A$ is circulant matrix.
\begin{lemma} \label{Slider}
    $A \in Circulant \Rightarrow$ $A_{i,j}=A_{i+k,j+k} \forall k \in \ZZ$ 
\end{lemma}
\begin{proof}
    Repeating the property $A_{i,j}=A_{i-1,j-1}$ together with 
    $a_{i,0}=a_{i-1,n-1}$\footnote{To make it work modulo $n$} $k \in \NN$
    times yields $A_{i,j}=A_{i-k,j-k}$
    then exchanging $(i-k,j-k) \leftrightarrow (i,j) \Rightarrow A_{i,j}=A_{i+k,j+k}$
    showing that it works for both directions and thus one can instead say $k \in \ZZ$
\end{proof}
\begin{lemma}
    $A \in $Circulant $\Rightarrow$
    $Ax = a \ast x $ \footnote{Circular convolution} where $a$ is the first colomn of $A$
    \end{lemma}
\begin{proof}
    $a \ast x = \{$Since convolution is commutative$\} = (x \ast a)_k \stackrel{def}{=} 
    \displaystyle\sum_{i=0}^{n-1}{x_i a_{k-i}} = \sum{x_i A_{k-i,0}} = \sum x_i A_{k,i} = \sum A_{k,i} x_i \stackrel{def}{=} (Ax)_k = Ax$
\end{proof}

We also have from fourier analysis the property:
$\Fourier\{c \ast x\}=\Fourier\{c\} \cdot \Fourier\{x\}$
\footnote{Where $\cdot$ is pointwise multiplication,
that is treating \Fourier\{x\} and \Fourier\{c\} as discrete functions.} 

The algorithm:
\begin{list}{*}{}
 \item Pickout $a$ from $A$ wich is $O(n)$
 \item Calculate $\hat{a}=\Fourier\{a\}$ and $\hat{x}=\Fourier\{x\}$ 
 with fft where both $a$ and $x$ is of length $n$, fft is known to have implementations running at $O(nlogn)$
 \item $\widehat{Ax} = \hat{a} \cdot \hat{b}$ still having the same length, pointwise multiplication runs at $O(n)$
 \item $Ax = \Fourier^{-1}\{\widehat{Ax}\}$ with ifft, similary runs in $O(nlogn)$ 
\end{list}

The algorithm gives us the vector $Ax$ in $O(nlogn)$ operations. $\qed$

\end{solution}
%\clearpage
%-----------------------
\begin{exercise}
{\bf
Show how to formulate the Vertex Cover problem as an integer linear program
(with a polynomial number of constraints).
}
\end{exercise}
\begin{solution}

Vertex Cover problem:
Find the minimal vertex cover $C$ in a graph $G$ with vertices $V$ and edges $E$

Define $\bar{x} \in \{0,1\}^n$ as 
\[
  x_i = \left\{
  \begin{array}{l l}
        1 & \quad \text{if vertex-$i \in C$ }\\
        0 & \quad \text{else}\\
  \end{array} \right.
\]
and $\tilde{E}$ as 
\[
    e_{ij} = \left\{
    \begin{array}{l l}
        1 & \quad \text{if there is an edge between vertex-$i$ and vertex-$j$}\\
        0 & \quad \text{else}\\
    \end{array} \right.
\]

$\tilde{E}^T=\tilde{E}$ (undirected graph)
$\bar{x} \ge \bar{0}$

With the constraint ``An edge of the graph has at least on vertex in the vertex-cover set connected to it''
$e_ij=1 => x_i=1 \verb+ OR + x_j=1$ or more mathematical notation $e_{ij}=1 \Rightarrow x_i + x_j \ge 1$ to get a lower amount of constraints. 

Define the vector $c = (1,1,..,1)$ with length $n$, representing that each vertice gets the same weight.\\
We want to minimize the number of vertices in the cover that is $\sum{x_i} = |\hat{x}|_1 = c^T x $

To sum up we have:\\
$min\{c^T x\}$ \\
with constraints $\{e_{ij}=1 \Rightarrow x_i + x_j \ge 1\}$\\
$x \in \{0,1\}^n$

Both the cost-function and constraints are linear and the domain is integer wich gives us a integer linear program.\\
Since $\tilde{E}$ is of size $O(n^2)$ the number of constraints is $O(n^2)$ thus polynomial number of linear constraints.

\end{solution}
%\clearpage
%-----------------------
\begin{exercise}
{\bf
Give an example of Euclidian TSP where the nearest neighbor heuristic fails
to find the optimal solution.
}
\end{exercise}
\begin{solution}

The nearest neighbor is a greedy algorithm and will sometimes
choose to walk to a city that would have been better to walk
to in a later stage, giving a shorter total walk.
\\
This simple example that gets the essence of the problem
of the greedy approach of the algorithm, here starting at city $A$
\footnote{Doesn't matter if we went to $D$ instead of $C$ from $B$ by symmetri-reason, just relabel $C \leftrightarrow D$}: 

\[\begin{matrix}
    \begin{tikzpicture}[>=stealth,->,shorten >=2pt,looseness=.5,auto]
        \node at (0,0) (A) {$A_{(0,0)}$};
        \node at (4,0) (B) {$B_{(4,0)}$};
        \node at (4,3) (C) {$C_{(4,3)}$};
        \node at (4,-3) (D) {$D_{(4,-3)}$};

        \draw (A) -- (B) node [midway] {4};
        \draw (B) -- (C) node [midway] {3};
        \draw (C) to[bend left] (D) node [midway] {6}; 
        \draw (D) -- (A) node [midway] {5};
    \end{tikzpicture}
    &
    \begin{tikzpicture}[>=stealth,->,shorten >=2pt,looseness=.5,auto]
        \node at (0,0) (A) {$A_{(0,0)}$};
        \node at (4,0) (B) {$B_{(4,0)}$};
        \node at (4,3) (C) {$C_{(4,3)}$};
        \node at (4,-3) (D) {$D_{(4,-3)}$};

        \draw (A) -- (C) node [midway] {5};
        \draw (C) -- (B) node [midway] {3};
        \draw (B) -- (D) node [midway] {3}; 
        \draw (D) -- (A) node [midway] {5};
    \end{tikzpicture}
    \\
    (NN)
    &
    (Better)
    \\
\end{matrix}\]


(NN): $\Sigma=18$ \\
(Better): $\Sigma=16$ \\

(NN) solution > (Better) solution \\
Showing that this example of Euclidian TSP
when starting at city $A$ has a non optimal solution since $\exists$ a better solution. $\qed$

\end{solution}
%\clearpage
%-----------------------
\begin{exercise}
{\bf
Generalize the result for nearest neighbor (NN) as follows.
Find a constant $K > 1$ and an increasing function $f(n)$ such that,
for each $n$ there is a Euclidian TSP instance with $f(n)$ cities for which NN,
starting in city $1$ will yield a solution that is at least $K$ times larger
than the optimum solution.
You get $2$ bonus points if $K$ is such that Christofides' algorithm is guaranteed
to do better on these instances.
}
\end{exercise}
\begin{solution}
Choosing $A$ as city ``1'' and constructing the TSP instances as below with $f(n)=\underline{2n+2}$ cities. We assume that $n>0$.

\[\begin{matrix}
    \begin{tikzpicture}[>=stealth,->,shorten >=1pt,looseness=.5,auto]
        \node at (-16/3,0) (A) {$A$};
        \node at (0,0) (B) {$B$};
        
        \node at (0,1) (C1) {$C^1$};
        \node at (0,2) (C2) {$C^2$};
        %...
        \node at (0,4) (C4) {$C^n$};

        \node at (0,-1) (D1) {$D^1$};
        \node at (0,-2) (D2) {$D^2$};
        %...
        \node at (0,-4) (D4) {$D^n$};


        \draw (A) -- (B) node [midway] {4n};
        
        \draw (B) -- (C1) node [midway] {3};
        \draw (C1) -- (C2) node [midway] {3};
        \draw[dashed] (C2) -- (C4) node [midway] {3(n-2)};

        \draw (C4) to[bend left] (D1) node [midway] {3(n+1)};

        \draw (D1) -- (D2) node [midway] {3};
        \draw[dashed] (D2) -- (D4) node [midway] {3(n-2)};

        \draw (D4) -- (A) node [midway] { $5n$ };

    \end{tikzpicture}
    &
    \begin{tikzpicture}[>=stealth,->,shorten >=1pt,looseness=.5,auto]
        \node at (-16/3,0) (A) {$A$};
        \node at (0,0) (B) {$B$};
        
        \node at (0,1) (C1) {$C^1$};
        \node at (0,2) (C2) {$C^2$};
        %...
        \node at (0,4) (C4) {$C^n$};

        \node at (0,-1) (D1) {$D^1$};
        \node at (0,-2) (D2) {$D^2$};
        %...
        \node at (0,-4) (D4) {$D^n$};


        \draw (A) -- (C4) node [midway] {5n};
        \draw[dashed] (C4) -- (C2) node [midway] {3(n-2)};
        \draw (C2) -- (C1) node [midway] {3};
        \draw (C1) -- (B) node [midway] {3};
        \draw (B) -- (D1) node [midway] {3};
        \draw (D1) -- (D2) node [midway] {3};
        \draw[dashed] (D2) -- (D4) node [midway] {3(n-2)};
        \draw (D4) -- (A) node [midway] { $5n$ };

    \end{tikzpicture}

    \\
    (NN)
    &
    (Better)
    \\
\end{matrix}\]

(NN): $\Sigma=4n+3n+3(n+1)+3(n-1)+5n=18n$\\
(Better): $\Sigma=5n+3n+3n+5n=16n$

(NN) $= K \cdot$ (Better) $\le$ K $\cdot$ (OPT)
$K=\frac{(NN)}{(Better)}=\frac{18n}{16n}=\underline{\frac{9}{8}}>1$

$B$ is always the closest city from $A$ \footnote{Doesn't matter if we went to $D^1$ instead of $C^1$ from $B$ by symmetri-reason just relabel $C^i \leftrightarrow D^i$}
The walk will always be $K=\frac{9}{8}$ worse then a better solution and therefore it is at least $\frac{9}{8}$ times worse then the optimal solution.

%Christofides' algorithm on the TSP instances: \\

\end{solution}
%\clearpage

\end{document}
